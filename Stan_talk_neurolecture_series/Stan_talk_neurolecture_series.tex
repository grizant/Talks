\PassOptionsToPackage{unicode=true}{hyperref} % options for packages loaded elsewhere
\PassOptionsToPackage{hyphens}{url}
%
\documentclass[ignorenonframetext,]{beamer}
\setbeamertemplate{caption}[numbered]
\setbeamertemplate{caption label separator}{: }
\setbeamercolor{caption name}{fg=normal text.fg}
\beamertemplatenavigationsymbolsempty
\usepackage{lmodern}
\usepackage{amssymb,amsmath}
\usepackage{ifxetex,ifluatex}
\usepackage{fixltx2e} % provides \textsubscript
\ifnum 0\ifxetex 1\fi\ifluatex 1\fi=0 % if pdftex
  \usepackage[T1]{fontenc}
  \usepackage[utf8]{inputenc}
  \usepackage{textcomp} % provides euro and other symbols
\else % if luatex or xelatex
  \usepackage{unicode-math}
  \defaultfontfeatures{Ligatures=TeX,Scale=MatchLowercase}
\fi
% use upquote if available, for straight quotes in verbatim environments
\IfFileExists{upquote.sty}{\usepackage{upquote}}{}
% use microtype if available
\IfFileExists{microtype.sty}{%
\usepackage[]{microtype}
\UseMicrotypeSet[protrusion]{basicmath} % disable protrusion for tt fonts
}{}
\IfFileExists{parskip.sty}{%
\usepackage{parskip}
}{% else
\setlength{\parindent}{0pt}
\setlength{\parskip}{6pt plus 2pt minus 1pt}
}
\usepackage{hyperref}
\hypersetup{
            pdftitle={Reasoning with Uncertainty the Bayesian way},
            pdfauthor={AG Schissler},
            pdfborder={0 0 0},
            breaklinks=true}
\urlstyle{same}  % don't use monospace font for urls
\newif\ifbibliography
\usepackage{color}
\usepackage{fancyvrb}
\newcommand{\VerbBar}{|}
\newcommand{\VERB}{\Verb[commandchars=\\\{\}]}
\DefineVerbatimEnvironment{Highlighting}{Verbatim}{commandchars=\\\{\}}
% Add ',fontsize=\small' for more characters per line
\usepackage{framed}
\definecolor{shadecolor}{RGB}{248,248,248}
\newenvironment{Shaded}{\begin{snugshade}}{\end{snugshade}}
\newcommand{\AlertTok}[1]{\textcolor[rgb]{0.94,0.16,0.16}{#1}}
\newcommand{\AnnotationTok}[1]{\textcolor[rgb]{0.56,0.35,0.01}{\textbf{\textit{#1}}}}
\newcommand{\AttributeTok}[1]{\textcolor[rgb]{0.77,0.63,0.00}{#1}}
\newcommand{\BaseNTok}[1]{\textcolor[rgb]{0.00,0.00,0.81}{#1}}
\newcommand{\BuiltInTok}[1]{#1}
\newcommand{\CharTok}[1]{\textcolor[rgb]{0.31,0.60,0.02}{#1}}
\newcommand{\CommentTok}[1]{\textcolor[rgb]{0.56,0.35,0.01}{\textit{#1}}}
\newcommand{\CommentVarTok}[1]{\textcolor[rgb]{0.56,0.35,0.01}{\textbf{\textit{#1}}}}
\newcommand{\ConstantTok}[1]{\textcolor[rgb]{0.00,0.00,0.00}{#1}}
\newcommand{\ControlFlowTok}[1]{\textcolor[rgb]{0.13,0.29,0.53}{\textbf{#1}}}
\newcommand{\DataTypeTok}[1]{\textcolor[rgb]{0.13,0.29,0.53}{#1}}
\newcommand{\DecValTok}[1]{\textcolor[rgb]{0.00,0.00,0.81}{#1}}
\newcommand{\DocumentationTok}[1]{\textcolor[rgb]{0.56,0.35,0.01}{\textbf{\textit{#1}}}}
\newcommand{\ErrorTok}[1]{\textcolor[rgb]{0.64,0.00,0.00}{\textbf{#1}}}
\newcommand{\ExtensionTok}[1]{#1}
\newcommand{\FloatTok}[1]{\textcolor[rgb]{0.00,0.00,0.81}{#1}}
\newcommand{\FunctionTok}[1]{\textcolor[rgb]{0.00,0.00,0.00}{#1}}
\newcommand{\ImportTok}[1]{#1}
\newcommand{\InformationTok}[1]{\textcolor[rgb]{0.56,0.35,0.01}{\textbf{\textit{#1}}}}
\newcommand{\KeywordTok}[1]{\textcolor[rgb]{0.13,0.29,0.53}{\textbf{#1}}}
\newcommand{\NormalTok}[1]{#1}
\newcommand{\OperatorTok}[1]{\textcolor[rgb]{0.81,0.36,0.00}{\textbf{#1}}}
\newcommand{\OtherTok}[1]{\textcolor[rgb]{0.56,0.35,0.01}{#1}}
\newcommand{\PreprocessorTok}[1]{\textcolor[rgb]{0.56,0.35,0.01}{\textit{#1}}}
\newcommand{\RegionMarkerTok}[1]{#1}
\newcommand{\SpecialCharTok}[1]{\textcolor[rgb]{0.00,0.00,0.00}{#1}}
\newcommand{\SpecialStringTok}[1]{\textcolor[rgb]{0.31,0.60,0.02}{#1}}
\newcommand{\StringTok}[1]{\textcolor[rgb]{0.31,0.60,0.02}{#1}}
\newcommand{\VariableTok}[1]{\textcolor[rgb]{0.00,0.00,0.00}{#1}}
\newcommand{\VerbatimStringTok}[1]{\textcolor[rgb]{0.31,0.60,0.02}{#1}}
\newcommand{\WarningTok}[1]{\textcolor[rgb]{0.56,0.35,0.01}{\textbf{\textit{#1}}}}
\usepackage{longtable,booktabs}
\usepackage{caption}
% These lines are needed to make table captions work with longtable:
\makeatletter
\def\fnum@table{\tablename~\thetable}
\makeatother
\usepackage{graphicx,grffile}
\makeatletter
\def\maxwidth{\ifdim\Gin@nat@width>\linewidth\linewidth\else\Gin@nat@width\fi}
\def\maxheight{\ifdim\Gin@nat@height>\textheight\textheight\else\Gin@nat@height\fi}
\makeatother
% Scale images if necessary, so that they will not overflow the page
% margins by default, and it is still possible to overwrite the defaults
% using explicit options in \includegraphics[width, height, ...]{}
\setkeys{Gin}{width=\maxwidth,height=\maxheight,keepaspectratio}
% Prevent slide breaks in the middle of a paragraph:
\widowpenalties 1 10000
\raggedbottom
\setbeamertemplate{part page}{
\centering
\begin{beamercolorbox}[sep=16pt,center]{part title}
  \usebeamerfont{part title}\insertpart\par
\end{beamercolorbox}
}
\setbeamertemplate{section page}{
\centering
\begin{beamercolorbox}[sep=12pt,center]{part title}
  \usebeamerfont{section title}\insertsection\par
\end{beamercolorbox}
}
\setbeamertemplate{subsection page}{
\centering
\begin{beamercolorbox}[sep=8pt,center]{part title}
  \usebeamerfont{subsection title}\insertsubsection\par
\end{beamercolorbox}
}
\AtBeginPart{
  \frame{\partpage}
}
\AtBeginSection{
  \ifbibliography
  \else
    \frame{\sectionpage}
  \fi
}
\AtBeginSubsection{
  \frame{\subsectionpage}
}
\setlength{\emergencystretch}{3em}  % prevent overfull lines
\providecommand{\tightlist}{%
  \setlength{\itemsep}{0pt}\setlength{\parskip}{0pt}}
\setcounter{secnumdepth}{0}

% set default figure placement to htbp
\makeatletter
\def\fps@figure{htbp}
\makeatother

\usepackage{graphicx}
\usepackage{pdfpages}

\title{Reasoning with Uncertainty the Bayesian way}
\providecommand{\subtitle}[1]{}
\subtitle{with examples in Cognitive Modeling in R and Stan}
\author{AG Schissler}
\date{2/09/2018}

\begin{document}
\frame{\titlepage}

\begin{frame}
\tableofcontents[hideallsubsections]
\end{frame}
\hypertarget{motivation}{%
\section{Motivation}\label{motivation}}

\begin{frame}{%
\protect\hypertarget{the-problem}{%
The problem}}

\begin{itemize}
\tightlist
\item
  Reproducibility crisis (what to cite?)
\item
  confusion
\item
  Impact of Statistical methods for research workers Fisher (1925)
\end{itemize}

\end{frame}

\begin{frame}{%
\protect\hypertarget{asa-on-p-values-from-wasserstein2016}{%
ASA on p-values from Wasserstein and Lazar (2016)}}

\begin{enumerate}
[1.]
\tightlist
\item
  P-values can indicate how incompatible the data are with a specified
  statistical model.
\item
  P-values do not measure the probability that the studied hypothesis is
  true, or the probability that the data were produced by random chance
  alone.
\item
  Scientific conclusions and business or policy decisions should not be
  based only on whether a p-value passes a specific threshold.
\item
  Proper inference requires full reporting and transparency
\item
  A p-value, or statistical significance, does not measure the size of
  an effect or the importance of a result.
\item
  By itself, a p-value does not provide a good measure of evidence
  regarding a model or hypothesis.
\end{enumerate}

\end{frame}

\begin{frame}{%
\protect\hypertarget{statistical-rethinking-mcelreath2016}{%
Statistical Rethinking McElreath (2016)}}

\includegraphics[page=4,width=0.8\paperwidth]{Lecture01.pdf}

\end{frame}

\begin{frame}{%
\protect\hypertarget{flowchart}{%
flowchart}}

\includegraphics[height=0.9\paperheight, keepaspectratio]{stats_flow_chart_v2014.pdf}

\end{frame}

\begin{frame}{%
\protect\hypertarget{lindley-1986-quote}{%
Lindley 1986 quote?}}

\end{frame}

\hypertarget{bayes-101}{%
\section{Bayes 101}\label{bayes-101}}

\begin{frame}{%
\protect\hypertarget{general-principles}{%
General principles}}

\end{frame}

\begin{frame}{%
\protect\hypertarget{prediction}{%
Prediction}}

\end{frame}

\begin{frame}{%
\protect\hypertarget{sequential-updating}{%
Sequential updating}}

\end{frame}

\begin{frame}{%
\protect\hypertarget{mcmc}{%
MCMC}}

\end{frame}

\hypertarget{case-study-i-inferring-iq-using-gaussians}{%
\section{Case Study I: Inferring IQ using
Gaussians}\label{case-study-i-inferring-iq-using-gaussians}}

\begin{frame}{%
\protect\hypertarget{i.1.-research-question-and-data}{%
I.1. Research question and data}}

We seek to estimate the IQ of a set of people. There are only three
subjects and each take three IQ tests. The data are displayed below:

\begin{longtable}[]{@{}lrrr@{}}
\toprule
& Measurement1 & Measurement2 & Measurement3\tabularnewline
\midrule
\endhead
Subject1 & 90 & 95 & 100\tabularnewline
Subject2 & 105 & 110 & 115\tabularnewline
Subject3 & 150 & 155 & 160\tabularnewline
\bottomrule
\end{longtable}

\end{frame}

\begin{frame}{%
\protect\hypertarget{i.2.-graphical-model}{%
I.2. Graphical model}}

\end{frame}

\begin{frame}[fragile]{%
\protect\hypertarget{i.3.-stan-model-code}{%
I.3. Stan model Code}}

\begin{Shaded}
\begin{Highlighting}[]
\NormalTok{model <-}\StringTok{ "}
\StringTok{// Repeated Measures of IQ}
\StringTok{data \{ }
\StringTok{  int<lower=1> n;}
\StringTok{  int<lower=1> m;}
\StringTok{  matrix[n, m] x;}
\StringTok{\}}
\StringTok{parameters \{}
\StringTok{  vector<lower=0,upper=300>[n] mu;}
\StringTok{  real<lower=0,upper=100> sigma;}
\StringTok{\} }
\StringTok{model \{}
\StringTok{  // Data Come From Gaussians With Different Means}
\StringTok{  // But Common Standard Deviation}
\StringTok{  for (i in 1:n)}
\StringTok{    for (j in 1:m)  }
\StringTok{      x[i,j] ~ normal(mu[i], sigma);}
\StringTok{\}"}
\end{Highlighting}
\end{Shaded}

\end{frame}

\begin{frame}[fragile]{%
\protect\hypertarget{i.3-code}{%
I.3 Code}}

\begin{Shaded}
\begin{Highlighting}[]
\NormalTok{n <-}\StringTok{ }\KeywordTok{nrow}\NormalTok{(x) }\CommentTok{# number of people}
\NormalTok{m <-}\StringTok{ }\KeywordTok{ncol}\NormalTok{(x) }\CommentTok{# number of repeated measurements}

\NormalTok{data <-}\StringTok{ }\KeywordTok{list}\NormalTok{(}\DataTypeTok{x=}\NormalTok{x, }\DataTypeTok{n=}\NormalTok{n, }\DataTypeTok{m=}\NormalTok{m) }\CommentTok{# to be passed on to Stan}
\NormalTok{myinits <-}\StringTok{ }\KeywordTok{list}\NormalTok{(}
  \KeywordTok{list}\NormalTok{(}\DataTypeTok{mu=}\KeywordTok{rep}\NormalTok{(}\DecValTok{100}\NormalTok{, n), }\DataTypeTok{sigma=}\DecValTok{1}\NormalTok{))}

\CommentTok{# parameters to be monitored: }
\NormalTok{parameters <-}\StringTok{ }\KeywordTok{c}\NormalTok{(}\StringTok{"mu"}\NormalTok{, }\StringTok{"sigma"}\NormalTok{)}
\end{Highlighting}
\end{Shaded}

\end{frame}

\begin{frame}[fragile]{%
\protect\hypertarget{i.3-code-1}{%
I.3 Code}}

\begin{Shaded}
\begin{Highlighting}[]
\CommentTok{# The following command calls Stan with specific options.}
\CommentTok{# For a detailed description type "?rstan".}
\NormalTok{samples <-}\StringTok{ }\KeywordTok{stan}\NormalTok{(}\DataTypeTok{model_code=}\NormalTok{model,   }
                \DataTypeTok{data=}\NormalTok{data, }
                \DataTypeTok{init=}\NormalTok{myinits,  }\CommentTok{# If not specified, gives random inits}
                \DataTypeTok{pars=}\NormalTok{parameters,}
                \DataTypeTok{iter=}\DecValTok{2000}\NormalTok{, }
                \DataTypeTok{chains=}\DecValTok{1}\NormalTok{, }
                \DataTypeTok{thin=}\DecValTok{1}\NormalTok{,}
                \CommentTok{# warmup = 100,  # Stands for burn-in; Default = iter/2}
                \CommentTok{# seed = 123  # Setting seed; Default is random seed}
\NormalTok{                )}
\CommentTok{# Now the values for the monitored parameters are in the "samples" object, }
\CommentTok{# ready for inspection.}
\end{Highlighting}
\end{Shaded}

\end{frame}

\begin{frame}[fragile]{%
\protect\hypertarget{i.4.-output-and-discussion}{%
I.4. Output and discussion}}

\begin{verbatim}
## Inference for Stan model: a960c093e4a9e0d2f6d99c1c2e93a893.
## 1 chains, each with iter=2000; warmup=1000; thin=1; 
## post-warmup draws per chain=1000, total post-warmup draws=1000.
## 
##         mean se_mean   sd   2.5%    25%    50%    75%  97.5% n_eff Rhat
## mu[1]  95.12    0.14 4.17  86.53  92.67  95.15  97.38 104.35   864    1
## mu[2] 109.88    0.16 4.02 102.17 107.59 109.98 112.33 117.81   611    1
## mu[3] 154.84    0.15 4.08 146.56 152.51 154.82 157.26 162.82   739    1
## sigma   6.56    0.16 2.57   3.56   4.82   5.96   7.46  13.27   273    1
## lp__   -5.87    0.15 1.95 -10.71  -7.05  -5.44  -4.35  -3.36   165    1
## 
## Samples were drawn using NUTS(diag_e) at Wed Feb  7 12:33:41 2018.
## For each parameter, n_eff is a crude measure of effective sample size,
## and Rhat is the potential scale reduction factor on split chains (at 
## convergence, Rhat=1).
\end{verbatim}

\end{frame}

\begin{frame}{%
\protect\hypertarget{i.5.-output-and-discussion}{%
I.5. Output and discussion}}

\includegraphics{Stan_talk_neurolecture_series_files/figure-beamer/IQ_output2-1.pdf}

\end{frame}

\begin{frame}{%
\protect\hypertarget{i.5.-extension-for-repeated-measures}{%
I.5. Extension for repeated measures}}

\end{frame}

\hypertarget{case-study-ii-hierachical-signal-detection}{%
\section{Case Study II: Hierachical signal
detection}\label{case-study-ii-hierachical-signal-detection}}

\begin{frame}{%
\protect\hypertarget{research-question-and-data}{%
Research question and data}}

\end{frame}

\begin{frame}{%
\protect\hypertarget{graphical-model}{%
Graphical model}}

\end{frame}

\begin{frame}{%
\protect\hypertarget{code}{%
Code}}

\end{frame}

\begin{frame}{%
\protect\hypertarget{output-and-discussion}{%
Output and discussion}}

\end{frame}

\hypertarget{case-study-iii-psychophysical}{%
\section{Case Study III:
Psychophysical}\label{case-study-iii-psychophysical}}

\begin{frame}{%
\protect\hypertarget{research-question-and-data-1}{%
Research question and data}}

\end{frame}

\begin{frame}{%
\protect\hypertarget{graphical-model-1}{%
Graphical model}}

\end{frame}

\begin{frame}{%
\protect\hypertarget{code-1}{%
Code}}

\end{frame}

\begin{frame}{%
\protect\hypertarget{output-and-discussion-1}{%
Output and discussion}}

\end{frame}

\begin{frame}{%
\protect\hypertarget{extension-to-containment}{%
Extension to containment}}

\end{frame}

\hypertarget{conclusion-and-references}{%
\section{Conclusion and references}\label{conclusion-and-references}}

\begin{frame}{%
\protect\hypertarget{take-home-points}{%
Take home points}}

\begin{itemize}
\tightlist
\item
  Bayesian modeling is natural and flexible
\item
  Tools exist to aeou
\item
  collaboration opportunities
\end{itemize}

\end{frame}

\begin{frame}{%
\protect\hypertarget{software-used}{%
Software used}}

\end{frame}

\begin{frame}{%
\protect\hypertarget{references}{%
References}}

\hypertarget{refs}{}
\leavevmode\hypertarget{ref-Fisher1925}{}%
Fisher, Ronald Aylmer. 1925. \emph{Statistical methods for research
workers}. Genesis Publishing Pvt Ltd.

\leavevmode\hypertarget{ref-McElreath2016}{}%
McElreath, Richard. 2016. \emph{Statistical Rethinking}.
\url{https://doi.org/10.3102/1076998616659752}.

\leavevmode\hypertarget{ref-Wasserstein2016}{}%
Wasserstein, Ronald L, and Nicole A Lazar. 2016. “The ASA’s Statement on
p-Values: Context, Process, and Purpose.” \emph{The American
Statistician} 70 (2):129–33.
\url{https://doi.org/10.1080/00031305.2016.1154108}.

\end{frame}

\end{document}
